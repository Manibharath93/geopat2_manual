A signature is a numerical description of a motifel.

\section{Cartesian product ({\it prod})}

This method calculates signature as a $k$-dimensional histogram using $k$ primitive features assigned to each cell. 
Examples of such features include cell class, the size of the clump to which the cell belongs, the shape of the clump and its spatial orientation (\cite{Williams2008}). 
Because all features must be categorical (so the histogram can be formed), numerical features need to be categorized. 
For example, clump sizes need to be categorized into size categories from the smallest to the largest.
The number of bins in the crossproduct histogram is $N_1 \times N_2 \times \ldots \times N_k$, where $N_i$ is the number of categories of $i$-th feature. 
Crossproduct signature is designed to be effective in encapsulating spatial structures with clear geometric quality (having relatively low complexity); an example of a dataset with such a structure is the land cover raster.

\section{Class co-occurrence histogram ({\it cooc})}

This method uses a color co-occurrence histogram (\cite{Barnsley1996,Chang1999}), a variant of the Gray-Level Co-occurrence Matrix (GLCM) originally introduced by \cite{Haralick1973} to characterize texture in grayscale images.
In GeoPAT, color is replaced by cell class and a single cell separation of one pixel is used to calculate a co-occurrence histogram. 
This results in a single primitive feature - a pair of classes assigned to two neighboring cells; four-connectivity is assumed for establishing the existence of a neighborhood relationship between the two cells. 
Thus, four features are calculated for each cell, but their total number is halved as the same feature is generated twice by the pairs of neighboring cells.
For a scene with $k$ cell classes, the co-occurrence histogram has $(k^2+k)/2$ bins, $k$ of them correspond to same-class pairs, which measure the composition of the classes in the scene, and $(k^2-k)/2$ bins correspond to different-class pairs, which measure the configuration of the classes in the scene. 
The co-occurrence signature is designed to be effective in encapsulating spatial structures exhibiting high complexity patterns like the ones resulting from a geomorphons-based classification of a DEM. 

\section{Decomposition histogram ({\it fdec})}

This signature method, inspired by the work of \cite{Remmel2006}, is used to describe a scene using a set of sub-scenes having a hierarchy of sizes. 
For the decomposition method to work the scene should be a square having a linear size of $2^{D}$ cells.
The scene with $k$ cell classes is scanned without overlap by a series of square moving windows with sizes $w=2^{i}$ cells where $i=2, \ldots, D$ are the decomposition levels. 
The size of the maximum scanning window, $2^D$, is the size of the scene. 
As a default, {\it fdec} uses the highest level of decomposition. 
It can be changed with the level parameter.
At each scanning position the percentages, ${p_1, \ldots, p_k}$, of the window's area occupied by cells having classes ${1, \ldots, k}$, respectively are recorded and a window area is assigned a list of $k$ tags (one for each class) representing those percentages. 
These tags are classified into one of three categories, 1 if the percentage is below $\frac{1}{4}$, 2 if it is between $\frac{1}{4}$ and $\frac{1}{2}$, and 3 if it is above $\frac{1}{2}$. 
Tallying all tags results in a histogram with 3$\times k$ bins (three bins for each class).  
The decomposition signature is designed to be effective for patterns of all levels of complexity, however, we have not yet accumulated sufficient experience working with this signature to offer definitive advice on the types of datasets to which it can be best applied. 

% \section{Local binary pattern histogram ({\it lbp})}
% 
% This feature is experimental.

\section{Landscape indices vector ({\it lind} and {\it linds})}

Two types of landscape indices are present in GeoPAT - landscape level indices and class level indices.
Each of landscape level indices produce one value for each "landscape" (basic unit of analysis - it could be a motifel or a polygon).
Class level indices produce as many values as there are classes in the input dataset.
For example, class percentage of landscape {\it pland} gives the proportional abundance of each class for each "landscape" in percentages.
Therefore, it produces one value for each class for each "landscape". 
The complete list of landscape indices supported by GeoPAT is in Table~\ref{lindtable}.

GeoPAT provides two ways to calculate landscape indices - {\it lind} (Landscape indices vector) and {\it linds} (Selected landscape indices vector).
The {\it lind} signature calculates all of the landscape level indices first, and all of the class level indices second.
Class level indices are arranged in such a way that the first index ({\it pland}) is calculated for all classes, then the second index ({\it lpi}) for all classes, etc. 
The {\it linds} signature calculates the landscape level indices and only one of the class level indices - class percentage of landscape {\it pland}. 
The reasoning for why using the aforementioned selected set of landscape indices ({\it linds}) for comparing landscapes is reasonable can be found in \cite{Niesterowicz2016EI}. 
This method is still experimental though.

% + Kolejność liczb w wynikowym pliku dla opcji "linds": Najpierw wszystkie indeksy landscape level w takiej kolejności jak w załączonym pliku csv, potem kompozycja klas w procentach, czyli np. dla NLCD będzie to wyglądać tak: 
% 
% lpi,ed,area_mn,area_am,area_md,area_ra,area_sd,area_cv,gyrate_mn,gyrate_am,gyrate_md,gyrate_ra,gyrate_sd,gyrate_cv,pafrac,para_mn,para_am,para_md,para_ra,para_sd,para_cv,shape_mn,shape_am,shape_md,shape_ra,shape_sd,shape_cv,frac_mn,frac_am,frac_md,frac_ra,frac_sd,frac_cv,contig_mn,contig_am,contig_md,contig_ra,contig_sd,contig_cv,contag,iji,pladj,ai,lsi,cohesion,pd,split,division,mesh,prd,rpr,shdi,sidi,msidi,shei,siei,msiei,pland11,pland12,pland21,pland22,pland23,pland24,pland31,pland41,pland42,pland43,pland52,pland71,pland81,pland82,pland90,pland95
% 
% + Kolejność liczb w wynikowym pliku dla opcji "lind": Najpierw wszystkie indeksy landscape level, jak wyżej, potem indeksy class level w taki sposób, że najpierw pierwszy indeks dla wszystkich klas, potem drugi indeks dla wszystkich klas itd. Na przykład:
% 
% lpi,ed,area_mn,...[reszta_landscape_level],[pland_dla_wszystkich_klas],[lpi_dla_wszystkich_klas],...
% 
% Czyli dla NLCD pełna lista indeksów przy opcji "lind" będzie następująca:
% 
% lpi,ed,area_mn,area_am,area_md,area_ra,area_sd,area_cv,gyrate_mn,gyrate_am,gyrate_md,gyrate_ra,gyrate_sd,gyrate_cv,pafrac,para_mn,para_am,para_md,para_ra,para_sd,para_cv,shape_mn,shape_am,shape_md,shape_ra,shape_sd,shape_cv,frac_mn,frac_am,frac_md,frac_ra,frac_sd,frac_cv,contig_mn,contig_am,contig_md,contig_ra,contig_sd,contig_cv,contag,iji,pladj,ai,lsi,cohesion,pd,split,division,mesh,prd,rpr,shdi,sidi,msidi,shei,siei,msiei,pland11,pland12,pland21,pland22,pland23,pland24,pland31,pland41,pland42,pland43,pland52,pland71,pland81,pland82,pland90,pland95,lpi11,lpi12,lpi21,lpi22,lpi23,lpi24,lpi31,lpi41,lpi42,lpi43,lpi52,lpi71,lpi81,lpi82,lpi90,lpi95,ed11,ed12,ed21,ed22,ed23,ed24,ed31,ed41,ed42,ed43,ed52,ed71,ed81,ed82,ed90,ed95,area_mn11,area_mn12,area_mn21,area_mn22,area_mn23,area_mn24,area_mn31,area_mn41,area_mn42,area_mn43,area_mn52,area_mn71,area_mn81,area_mn82,area_mn90,area_mn95,area_am11,area_am12,area_am21,area_am22,area_am23,area_am24,area_am31,area_am41,area_am42,area_am43,area_am52,area_am71,area_am81,area_am82,area_am90,area_am95,area_md11,area_md12,area_md21,area_md22,area_md23,area_md24,area_md31,area_md41,area_md42,area_md43,area_md52,area_md71,area_md81,area_md82,area_md90,area_md95,area_ra11,area_ra12,area_ra21,area_ra22,area_ra23,area_ra24,area_ra31,area_ra41,area_ra42,area_ra43,area_ra52,area_ra71,area_ra81,area_ra82,area_ra90,area_ra95,area_sd11,area_sd12,area_sd21,area_sd22,area_sd23,area_sd24,area_sd31,area_sd41,area_sd42,area_sd43,area_sd52,area_sd71,area_sd81,area_sd82,area_sd90,area_sd95,area_cv11,area_cv12,area_cv21,area_cv22,area_cv23,area_cv24,area_cv31,area_cv41,area_cv42,area_cv43,area_cv52,area_cv71,area_cv81,area_cv82,area_cv90,area_cv95,gyrat_mn11,gyrat_mn12,gyrat_mn21,gyrat_mn22,gyrat_mn23,gyrat_mn24,gyrat_mn31,gyrat_mn41,gyrat_mn42,gyrat_mn43,gyrat_mn52,gyrat_mn71,gyrat_mn81,gyrat_mn82,gyrat_mn90,gyrat_mn95,gyrat_am11,gyrat_am12,gyrat_am21,gyrat_am22,gyrat_am23,gyrat_am24,gyrat_am31,gyrat_am41,gyrat_am42,gyrat_am43,gyrat_am52,gyrat_am71,gyrat_am81,gyrat_am82,gyrat_am90,gyrat_am95,gyrat_md11,gyrat_md12,gyrat_md21,gyrat_md22,gyrat_md23,gyrat_md24,gyrat_md31,gyrat_md41,gyrat_md42,gyrat_md43,gyrat_md52,gyrat_md71,gyrat_md81,gyrat_md82,gyrat_md90,gyrat_md95,gyrat_ra11,gyrat_ra12,gyrat_ra21,gyrat_ra22,gyrat_ra23,gyrat_ra24,gyrat_ra31,gyrat_ra41,gyrat_ra42,gyrat_ra43,gyrat_ra52,gyrat_ra71,gyrat_ra81,gyrat_ra82,gyrat_ra90,gyrat_ra95,gyrat_sd11,gyrat_sd12,gyrat_sd21,gyrat_sd22,gyrat_sd23,gyrat_sd24,gyrat_sd31,gyrat_sd41,gyrat_sd42,gyrat_sd43,gyrat_sd52,gyrat_sd71,gyrat_sd81,gyrat_sd82,gyrat_sd90,gyrat_sd95,gyrat_cv11,gyrat_cv12,gyrat_cv21,gyrat_cv22,gyrat_cv23,gyrat_cv24,gyrat_cv31,gyrat_cv41,gyrat_cv42,gyrat_cv43,gyrat_cv52,gyrat_cv71,gyrat_cv81,gyrat_cv82,gyrat_cv90,gyrat_cv95,pafrac11,pafrac12,pafrac21,pafrac22,pafrac23,pafrac24,pafrac31,pafrac41,pafrac42,pafrac43,pafrac52,pafrac71,pafrac81,pafrac82,pafrac90,pafrac95,para_mn11,para_mn12,para_mn21,para_mn22,para_mn23,para_mn24,para_mn31,para_mn41,para_mn42,para_mn43,para_mn52,para_mn71,para_mn81,para_mn82,para_mn90,para_mn95,para_am11,para_am12,para_am21,para_am22,para_am23,para_am24,para_am31,para_am41,para_am42,para_am43,para_am52,para_am71,para_am81,para_am82,para_am90,para_am95,para_md11,para_md12,para_md21,para_md22,para_md23,para_md24,para_md31,para_md41,para_md42,para_md43,para_md52,para_md71,para_md81,para_md82,para_md90,para_md95,para_ra11,para_ra12,para_ra21,para_ra22,para_ra23,para_ra24,para_ra31,para_ra41,para_ra42,para_ra43,para_ra52,para_ra71,para_ra81,para_ra82,para_ra90,para_ra95,para_sd11,para_sd12,para_sd21,para_sd22,para_sd23,para_sd24,para_sd31,para_sd41,para_sd42,para_sd43,para_sd52,para_sd71,para_sd81,para_sd82,para_sd90,para_sd95,para_cv11,para_cv12,para_cv21,para_cv22,para_cv23,para_cv24,para_cv31,para_cv41,para_cv42,para_cv43,para_cv52,para_cv71,para_cv81,para_cv82,para_cv90,para_cv95,shape_mn11,shape_mn12,shape_mn21,shape_mn22,shape_mn23,shape_mn24,shape_mn31,shape_mn41,shape_mn42,shape_mn43,shape_mn52,shape_mn71,shape_mn81,shape_mn82,shape_mn90,shape_mn95,shape_am11,shape_am12,shape_am21,shape_am22,shape_am23,shape_am24,shape_am31,shape_am41,shape_am42,shape_am43,shape_am52,shape_am71,shape_am81,shape_am82,shape_am90,shape_am95,shape_md11,shape_md12,shape_md21,shape_md22,shape_md23,shape_md24,shape_md31,shape_md41,shape_md42,shape_md43,shape_md52,shape_md71,shape_md81,shape_md82,shape_md90,shape_md95,shape_ra11,shape_ra12,shape_ra21,shape_ra22,shape_ra23,shape_ra24,shape_ra31,shape_ra41,shape_ra42,shape_ra43,shape_ra52,shape_ra71,shape_ra81,shape_ra82,shape_ra90,shape_ra95,shape_sd11,shape_sd12,shape_sd21,shape_sd22,shape_sd23,shape_sd24,shape_sd31,shape_sd41,shape_sd42,shape_sd43,shape_sd52,shape_sd71,shape_sd81,shape_sd82,shape_sd90,shape_sd95,shape_cv11,shape_cv12,shape_cv21,shape_cv22,shape_cv23,shape_cv24,shape_cv31,shape_cv41,shape_cv42,shape_cv43,shape_cv52,shape_cv71,shape_cv81,shape_cv82,shape_cv90,shape_cv95,frac_mn11,frac_mn12,frac_mn21,frac_mn22,frac_mn23,frac_mn24,frac_mn31,frac_mn41,frac_mn42,frac_mn43,frac_mn52,frac_mn71,frac_mn81,frac_mn82,frac_mn90,frac_mn95,frac_am11,frac_am12,frac_am21,frac_am22,frac_am23,frac_am24,frac_am31,frac_am41,frac_am42,frac_am43,frac_am52,frac_am71,frac_am81,frac_am82,frac_am90,frac_am95,frac_md11,frac_md12,frac_md21,frac_md22,frac_md23,frac_md24,frac_md31,frac_md41,frac_md42,frac_md43,frac_md52,frac_md71,frac_md81,frac_md82,frac_md90,frac_md95,frac_ra11,frac_ra12,frac_ra21,frac_ra22,frac_ra23,frac_ra24,frac_ra31,frac_ra41,frac_ra42,frac_ra43,frac_ra52,frac_ra71,frac_ra81,frac_ra82,frac_ra90,frac_ra95,frac_sd11,frac_sd12,frac_sd21,frac_sd22,frac_sd23,frac_sd24,frac_sd31,frac_sd41,frac_sd42,frac_sd43,frac_sd52,frac_sd71,frac_sd81,frac_sd82,frac_sd90,frac_sd95,frac_cv11,frac_cv12,frac_cv21,frac_cv22,frac_cv23,frac_cv24,frac_cv31,frac_cv41,frac_cv42,frac_cv43,frac_cv52,frac_cv71,frac_cv81,frac_cv82,frac_cv90,frac_cv95,conti_mn11,conti_mn12,conti_mn21,conti_mn22,conti_mn23,conti_mn24,conti_mn31,conti_mn41,conti_mn42,conti_mn43,conti_mn52,conti_mn71,conti_mn81,conti_mn82,conti_mn90,conti_mn95,conti_am11,conti_am12,conti_am21,conti_am22,conti_am23,conti_am24,conti_am31,conti_am41,conti_am42,conti_am43,conti_am52,conti_am71,conti_am81,conti_am82,conti_am90,conti_am95,conti_md11,conti_md12,conti_md21,conti_md22,conti_md23,conti_md24,conti_md31,conti_md41,conti_md42,conti_md43,conti_md52,conti_md71,conti_md81,conti_md82,conti_md90,conti_md95,conti_ra11,conti_ra12,conti_ra21,conti_ra22,conti_ra23,conti_ra24,conti_ra31,conti_ra41,conti_ra42,conti_ra43,conti_ra52,conti_ra71,conti_ra81,conti_ra82,conti_ra90,conti_ra95,conti_sd11,conti_sd12,conti_sd21,conti_sd22,conti_sd23,conti_sd24,conti_sd31,conti_sd41,conti_sd42,conti_sd43,conti_sd52,conti_sd71,conti_sd81,conti_sd82,conti_sd90,conti_sd95,conti_cv11,conti_cv12,conti_cv21,conti_cv22,conti_cv23,conti_cv24,conti_cv31,conti_cv41,conti_cv42,conti_cv43,conti_cv52,conti_cv71,conti_cv81,conti_cv82,conti_cv90,conti_cv95,iji11,iji12,iji21,iji22,iji23,iji24,iji31,iji41,iji42,iji43,iji52,iji71,iji81,iji82,iji90,iji95,pladj11,pladj12,pladj21,pladj22,pladj23,pladj24,pladj31,pladj41,pladj42,pladj43,pladj52,pladj71,pladj81,pladj82,pladj90,pladj95,ai11,ai12,ai21,ai22,ai23,ai24,ai31,ai41,ai42,ai43,ai52,ai71,ai81,ai82,ai90,ai95,lsi11,lsi12,lsi21,lsi22,lsi23,lsi24,lsi31,lsi41,lsi42,lsi43,lsi52,lsi71,lsi81,lsi82,lsi90,lsi95,cohesion11,cohesion12,cohesion21,cohesion22,cohesion23,cohesion24,cohesion31,cohesion41,cohesion42,cohesion43,cohesion52,cohesion71,cohesion81,cohesion82,cohesion90,cohesion95,pd11,pd12,pd21,pd22,pd23,pd24,pd31,pd41,pd42,pd43,pd52,pd71,pd81,pd82,pd90,pd95,split11,split12,split21,split22,split23,split24,split31,split41,split42,split43,split52,split71,split81,split82,split90,split95,division11,division12,division21,division22,division23,division24,division31,division41,division42,division43,division52,division71,division81,division82,division90,division95,mesh11,mesh12,mesh21,mesh22,mesh23,mesh24,mesh31,mesh41,mesh42,mesh43,mesh52,mesh71,mesh81,mesh82,mesh90,mesh95

% latex table generated in R 3.4.1 by xtable 1.8-2 package
% Mon Oct 23 11:04:05 2017
\begin{longtable}{rllrr}
  \hline
ID & Name & Description & Landscape level & Class level \\ 
   1 & pland & class percentage of landscape &   0 &   1 \\ 
    2 & lpi & largest patch index &   1 &   1 \\ 
    3 & ed & edge density &   1 &   1 \\ 
    4 & area\_mn & patch area mean &   1 &   1 \\ 
    5 & area\_am & patch area area-weighted mean &   1 &   1 \\ 
    6 & area\_md & patch area median &   1 &   1 \\ 
    7 & area\_ra & patch area range &   1 &   1 \\ 
    8 & area\_sd & patch area standard deviation &   1 &   1 \\ 
    9 & area\_cv & patch area coeff. of variation &   1 &   1 \\ 
   10 & gyrate\_mn & radius of gyration mean &   1 &   1 \\ 
   11 & gyrate\_am & radius of gyration area-weighted mean &   1 &   1 \\ 
   12 & gyrate\_md & radius of gyration median &   1 &   1 \\ 
   13 & gyrate\_ra & radius of gyration range &   1 &   1 \\ 
   14 & gyrate\_sd & radius of gyration standard deviation &   1 &   1 \\ 
   15 & gyrate\_cv & radius of gyration coeff. of variation &   1 &   1 \\ 
   16 & pafrac & perimeter-area fractal dimension &   1 &   1 \\ 
   17 & para\_mn & perimeter-area ratio mean &   1 &   1 \\ 
   18 & para\_am & perimeter-area ratio area-weighted mean &   1 &   1 \\ 
   19 & para\_md & perimeter-area ratio median &   1 &   1 \\ 
   20 & para\_ra & perimeter-area ratio range &   1 &   1 \\ 
   21 & para\_sd & perimeter-area ratio standard deviation &   1 &   1 \\ 
   22 & para\_cv & perimeter-area ratio coeff. of variation &   1 &   1 \\ 
   23 & shape\_mn & shape index mean &   1 &   1 \\ 
   24 & shape\_am & shape index area-weighted mean &   1 &   1 \\ 
   25 & shape\_md & shape index median &   1 &   1 \\ 
   26 & shape\_ra & shape index range &   1 &   1 \\ 
   27 & shape\_sd & shape index standard deviation &   1 &   1 \\ 
   28 & shape\_cv & shape index coeff. of variation &   1 &   1 \\ 
   29 & frac\_mn & fractal index mean &   1 &   1 \\ 
   30 & frac\_am & fractal index area-weighted mean &   1 &   1 \\ 
   31 & frac\_md & fractal index median &   1 &   1 \\ 
   32 & frac\_ra & fractal index range &   1 &   1 \\ 
   33 & frac\_sd & fractal index standard deviation &   1 &   1 \\ 
   34 & frac\_cv & fractal index coeff. of variation &   1 &   1 \\ 
   35 & contig\_mn & contiguity index mean &   1 &   1 \\ 
   36 & contig\_am & contiguity index area-weighted mean &   1 &   1 \\ 
   37 & contig\_md & contiguity index median &   1 &   1 \\ 
   38 & contig\_ra & contiguity index range &   1 &   1 \\ 
   39 & contig\_sd & contiguity index standard deviation &   1 &   1 \\ 
   40 & contig\_cv & contiguity index coeff. of variation &   1 &   1 \\ 
   41 & contag & contagion index &   1 &   0 \\ 
   42 & iji & interspersion \& juxtaposition index &   1 &   1 \\ 
   43 & pladj & percentage of like adjacencies &   1 &   1 \\ 
   44 & ai & aggregation index &   1 &   1 \\ 
   45 & lsi & landscape shape index &   1 &   1 \\ 
   46 & cohesion & patch cohesion index &   1 &   1 \\ 
   47 & pd & patch density &   1 &   1 \\ 
   48 & split & splitting index &   1 &   1 \\ 
   49 & division & landscape division index &   1 &   1 \\ 
   50 & mesh & effective mesh size &   1 &   1 \\ 
   51 & prd & patch richness density &   1 &   0 \\ 
   52 & rpr & relative patch richness &   1 &   0 \\ 
   53 & shdi & Shannon's diversity index &   1 &   0 \\ 
   54 & sidi & Simpson's diversity index &   1 &   0 \\ 
   55 & msidi & modified Simpson's diversity index &   1 &   0 \\ 
   56 & shei & Shannon's evenness index &   1 &   0 \\ 
   57 & siei & Simpson's evenness index &   1 &   0 \\ 
   58 & msiei & modified Simpson's evenness index &   1 &   0 \\ 
  \hline
\caption{Landscape metrics implemented in GeoPAT 2.0} 
\label{lindtable}
\end{longtable}


More information about landscape indices can be found in the FRAGSTATS documentation (\cite{mcgarigal2014fragstats}).

% \section{J-Coocurrence vector ({\it jcov})}
% 
% This feature is experimental.

\section{Shannon entropy ({\it ent})}

This signature contains three values:

\begin{enumerate}
  \item Shannon entropy
  \item Number of unique values in a motifel/segment
  \item Size of a motifel/segment
\end{enumerate}

This signature was not designed for purposes of geoprocessing (searching, comparision, segmentation, clustering), but rather as a diagnostic tool.