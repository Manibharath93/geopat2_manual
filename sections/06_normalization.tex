There are two ways to normalize pattern signatures.
The first one (used in {\tt gpat\_gridhis}, {\tt gpat\_pointshis} and {\tt gpat\_polygon}) works independently on each of the pattern signatures.
The role of the second one (used in {\tt gpat\_globnorm} is to normalize each of the signature elements for all the grid. 
Decision on the normalization method and its type should depends on both, numerical signatures and dissimilarity measures used.
For example, the class co-occurrence histogram signature should be locally normalized (used in {\tt gpat\_gridhis}, {\tt gpat\_pointshis} and {\tt gpat\_polygon}) before a segmentation using the Jensen Shannon Divergence measure.
However, in case of a segmentation based on the landscape indices a global normalization {\tt gpat\_globnorm} is recommended, while a local normalization should be avoided.

\section{01}

Normalization to the [0,1] interval.

\section{pdf}

Normalization to pdf (sum(hi) = 1).

\section{N01}

Normalization to N(0,1) (hi = (hi - avg) / std).

\section{ind01}

Normalization to [0,1] for 72 landscape indices.
Used in the {\tt gpat\_globnorm} module only.
It accepts only the grids built using the selected landscape indices vector signature ({\it linds}).

\section{none}

Normalization process is skipped. 
