\section{System requirements}

The Windows installer and Linux binaries provide all necessary components. 
To build GeoPAT 2 from the source code, the user needs to install the GDAL library.

\section{Windows installer}
The Windows installer works under 64 bit versions of Windows 7, 8.1, and 10.
The installer provides four components:
\begin{itemize}
  \item{GDAL library package}
  \item{ezGDAL and SML libraries}
  \item{GeoPAT 2 software}
  \item{Microsoft Visual C 13 runtime libraries (optional)}
\end{itemize}
To start the installation process user has to run GPAT2setup.exe.
The setup program should be run in "Run as administrator" mode.
If an antivirus software is running on the computer, user should turn it off temporary for the time of GeoPAT installation.
The installer will create the directory for GDAL and GeoPAT and copy all necessary files.
Optionally, GPAT2setup.exe will start the Microsoft Visual C runtime libraries installer.
When installation is completed, the user can find "SIL GeoPAT 2" in the Windows start menu with two files - "GeoPAT console" and "Uninstall GeoPAT".

The installer for Windows x64 is available at:

\begin{lstlisting}[escapechar=|]
|\url{http://sil.uc.edu/cms/data/uploads/software_data/GPAT2/GPAT2setup.exe}|
\end{lstlisting}

\section{Building from source code}

The building from source code procedure is presented using the Fedora 27 Linux distribution.
This procedure could differ between Linux distributions, therefore the user should modify it to their system (see the travis file for the installation procedure for Ubuntu - \url{https://github.com/Nowosad/geopat2/blob/master/.travis.yml}).

To build GeoPAT 2 from the source, the user has to do the two following steps:

\begin{enumerate}
    \item{Install the developement version of GDAL}
    \item{Build and install the GeoPAT 2 software}
\end{enumerate}

The dnf package manager can be used in a Fedora system to install the development version of GDAL:

\begin{lstlisting}
dnf install gdal-devel
\end{lstlisting}

The second step of the installation procedure is a compilation of the GeoPAT 2 source code.
The source code of GeoPAT 2 is available at:
\begin{lstlisting}[escapechar=|]
|\url{https://github.com/Nowosad/geopat2/archive/master.zip}|
\end{lstlisting}
GeoPAT 2 depends on the GDAL, SML, and ezGDAL libraries (SML and ezGDAL are parts of GeoPAT 2).
Therefore, user has to unpack, compile and install the GeoPAT 2.
\begin{lstlisting}
make
make install
\end{lstlisting}
The "make" and "make install" commands should be run in the main GeoPAT 2 source code directory.

By default, the library is placed in the /usr/local/lib directory and the include file is placed in /usr/local/include.
The user can change the destination directory by adding the PREFIX parameter.
\begin{lstlisting}
make PREFIX=/my/destination/directory
\end{lstlisting}
When PREFIX is provided, the library is placed in /my/destination/directory/lib and the include file is placed in /my/destination/directory/include.
